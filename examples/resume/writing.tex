%-------------------------------------------------------------------------------
%	SECTION TITLE
%-------------------------------------------------------------------------------
\cvsection{写作}


%-------------------------------------------------------------------------------
%	CONTENT
%-------------------------------------------------------------------------------
\begin{cventries}

%---------------------------------------------------------
  \cventry
  {大停滞} % Title
  {彼特·蒂尔} % Role
    {} % Location
    {} % Date(s)
    {
      \begin{cvitems} % Description(s)
        \item {彼特·蒂尔是PayPal创始人,Facebook第一位外部投资人,《从0到1》作者。本文翻译自他和哈佛大学数学博士,家族基金高管埃里克·维因斯坦的对话,是其对科技停滞最完整的表述,共将近两万字}
        \item {主要观点是除计算机领域以外,科技进步在过去50年里相对停滞。交通工具速度停滞,飞机的速度甚至在变慢;人类预期寿命停滞,1970年以前,西方国家预期寿命每10年提升2.5岁,美国人预期寿命实际开始减少;航天停滞,1972年以后再无人登月;能源发展停滞,1973年石油危机之前,从木材到煤炭到石油,能源利用成本一直在减小,但再没探索出新能源}
        \item {围绕科技停滞,彼特·蒂尔对气候变化、政治、经济、教育等问题做出严肃批评。彼特·蒂尔本科就读于斯坦福哲学系,师从哲学家让内·吉拉德。投资Facebook决定是根据吉拉德的模仿理论做出}
        \item {为解决停滞问题,彼特·蒂尔成立了创始人基金,资助航空和交通、生物科技、高端硬件和软件、能源和互联网等困难领域,该基金投资了SpaceX和Airbnb等公司}
        \item {出于对经济增长的怀疑。彼特·蒂尔旗下对冲基金曾在金融危机期间大举做空美元和美国股市,但实际上金融危机后股市止跌回弹,最终损失惨重以致于关闭基金}
        \item {他在过去十年关于停滞进行过多次辩论,辩论者包括马克·安德森,埃里克·施密特(Google首任CEO)等等}
        \item {前美国财政部长,哈佛大学校长拉里·萨姆尔森也在认为表示未来经济将陷入长期停滞。发达国家联合起来相当于封闭经济体,储蓄等于投资。而储蓄倾向上升,投资需求下降,会导致未来利率降低,各种资产组合回报不足和现行政策不可延续}
      \end{cvitems}
    }

    \cventry
    {软件} % Title·
    {马克·安德森} % Role
    {} % Location
    {} % Date(s)
    {
      \begin{cvitems} % Description(s)
        \item {为什么软件正在吞噬这个世界 \ \ \ 原作者马克·安德森是第一个浏览器Mosaic作者}
        \item {此刻——创造之时:马克·安德森}
        \item {迁移到云:AWS CEO安迪·杰西访谈}
      \end{cvitems}
    }


    \cventry
    {经济学、统计学专业文章} % Title
    {其他文章} % Role
    {} % Location
    {} % Date(s)
    {
      \begin{cvitems} % Description(s)
        \item {P值的12中错误用法}
      \end{cvitems}
    }

%---------------------------------------------------------
\end{cventries}
