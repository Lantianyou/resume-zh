%-------------------------------------------------------------------------------
%	SECTION TITLE
%-------------------------------------------------------------------------------
\cvsection{项目}


%-------------------------------------------------------------------------------
%	CONTENT
%-------------------------------------------------------------------------------
\begin{cventries}

%---------------------------------------------------------
  \cventry
    {React Next.js} % Job title
    {个人网站} % Organization
    {} % Location
    {} % Date(s)
    {
      \begin{cvitems} % Description(s) of tasks/responsibilities
        \item {用React Context实现深色模式}
        \item {实现浏览式进度条、淡入淡出动画和响应式设计}
        \item {部署在Netlify上}
        \item {熟练使用VS Code git 等编程工具}
      \end{cvitems}
    }

%---------------------------------------------------------

%---------------------------------------------------------
\cventry
{全国一等奖} % Job title
{狮子金融期货交易大赛} % Organization
{} % Location
{} % Date(s)
{
  \begin{cvitems} % Description(s) of tasks/responsibilities
    \item {在特朗普当选后交易纳斯达克和标普500期权实现4218\%收益}
  \end{cvitems}
  }
%---------------------------------------------------------
%---------------------------------------------------------

\cventry
  {全国一等奖} % Job title
  {狮子金融期货交易大赛} % Organization
  {} % Location
  {} % Date(s)
{
  \begin{cvitems} % Description(s) of tasks/responsibilities
    \item {在特朗普当选后交易纳斯达克和标普500期权实现4218\%收益}
  \end{cvitems}
  }
%---------------------------------------------------------
%---------------------------------------------------------
\end{cventries}


% \begin{cventries}

%   %---------------------------------------------------------
%     \cventry
%       {Next.js Redux CSS} % Job title
%       {电商应用} % Organization
%       {} % Location
%       {} % Date(s)
%       {
%         \begin{cvitems} % Description(s) of tasks/responsibilities
%           \item {独立开发了一个电商网站,具有注册,登陆,创建删除商品,购买支付等功能}
%           \item {利用Next.JS 实现了全局布局设计,让导航栏显示加载时进度条,利用CSS实现购物车的折叠和弹出,用Jest简单测试,主要难点在于全部商品页的分页}
%           \item {独立设计GraphQL模式以便前后端分离开发,利用Stripe实现支付,整个数据库用Docker容器化}
%         \end{cvitems}
%       }
  
  %---------------------------------------------------------
  % \end{cventries}
